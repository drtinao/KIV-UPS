\documentclass[12pt, a4paper, pdftex, czech, titlepage]{report}

\usepackage[czech]{babel}
\usepackage[utf8]{inputenc}
\usepackage{lmodern}
\usepackage{textcomp}
\usepackage[T1]{fontenc}
\usepackage{amsfonts}
\usepackage{titlesec}
\usepackage{graphicx}
\usepackage{indentfirst}
\usepackage{grffile}
\usepackage{listings}
\usepackage{xcolor}
\usepackage{hyperref}
\lstset{
  language=C,               
  numbers=left,              
  stepnumber=1,                         
  numbersep=5pt,                  
  backgroundcolor=\color{white},  
  showspaces=false,              
  showstringspaces=false,       
  showtabs=false,                
  tabsize=2,                    
  captionpos=b,               
  breaklines=true,           
  breakatwhitespace=true,         
  title=\lstname    
}

\titleformat{\chapter}
  {\normalfont\LARGE\bfseries}{\thechapter}{1em}{}
\titlespacing*{\chapter}{0pt}{0ex plus 1ex minus .2ex}{2.0ex plus .2ex}

\begin{document}

\begin{titlepage}
	\vspace*{-2cm}
	{\centering\includegraphics[width=\textwidth]{logo.jpg}\par}
	\centering
	\vspace*{2cm}
	{\Large Semestrální práce - hra pro více hráčů (Prší)\par}
	\vspace{1.5cm}
	{\Huge\bfseries Úvod do počítačových sítí\par}
	\vspace{7cm}

		
	{\Large Ondřej Drtina\par}
	{\Large A17B0202P\par}
	{\Large drtinao@students.zcu.cz\par}
	\vfill
\end{titlepage}

\tableofcontents
\thispagestyle{empty}
\clearpage

\chapter{Zadání}
Cílem semestrální práce v rámci předmětu UPS bylo vytvořit síťovou hru pro více hráčů.
Konkrétní hru si mohl student po konzultaci s příslušným cvičícím zvolit,
avšak i přesto byla stanovena určitá pravidla pro její tvorbu. Vzhledem k tomu,
že práce zahrnuje tvorbu herního serveru i klienta, lze požadavky rozdělit do dvou
sekcí.
\section{Požadavky na server}
Herní server musí splňovat zejména následující požadavky:

\begin{itemize}
\item musí být napsán v nízkoúrovňovém programovacím jazyce (C / C++)
\item je schopný obsluhy více klientů současně
\item možnost znovu navázet komunikaci s klientem po výpadku spojení a umožnění pokračování ve hře
\end{itemize}

\section{Požadavky na klienta}
Herní klient musí splňovat následující požadavky:

\begin{itemize}
\item psán ve vysokoúrovňovém programovacím jazyce (Java / Unity)
\item disponování grafickým uživatelským rozhraním
\item kompatibilita s Windows i Linux
\item možnost práce s herními místnostmi (lobby)

\end{itemize}

\section{Společné požadavky}
Některé z požadavků jsou shodné pro serverovou i klientskou část aplikace, například:

\begin{itemize}
\item komunikace pomocí protokolu TCP (bez zabezpečení)
\item řešení výpadků komunikace (oznámení a patřičná reakce při odpojení klienta / serveru)
\item možnost automatického překladu (make, Ant, Maven)
\item odolnost vůči nevalidním zprávám
\end{itemize}

\section{Celé znění úlohy}
Celé zadání úlohy je možno nalézt online \href{https://courseware.zcu.cz/portal/studium/courseware/kiv/ups/samostatna-prace.html}{zde}.
\vspace{0,5 cm}

\chapter{Analýza problému}
Řešení úlohy vyžadovalo především nastudování způsobu komunikace pomocí
soketů. Následně pak bylo nutno vyhodnotit možné způsoby implementace tohoto typu komunikace
v rámci jazyků, jež byly použity pro tvorbu hry (v mém případě Java a C). Další
kroky učiněné v rámci analýzy lze rozdělit do dvou kategorií - viz podkapitoly).

\section{Herní klient}
Tvorba herního klienta nevyžadovala příliš analytické činnosti, jelikož jsem se klienta
rozhodl vytvořit v jazyce Java (s vyžitím knihoven AWT / Swing). Zmíněný
programovací jazyk jsem již pro pár prací použil a jsem tedy obeznámen se
syntaxí běžně užívaných příkazů tohoto jazyka. Zřejmě jediné možné úskalí 
jsem tedy v době analýzy viděl pouze v komunikaci prostřednictvím soketů.
Rychle se však ukázalo, že v případě Javy je řešení síťové komunikace poměrně jednoduchou záležitostí
i pro začátečníka a vyžaduje minimální režii ze strany programátora.

\section{Herní server}
Ve srovnání s herním klientem vyžadovala tvorba herního serveru o poznání delší čas strávený
analýzou. Tento fakt přikládám zejména skutečnosti, že herní server musel být vytvořen
v nízkoúrovňovém jazyce (např. C), se kterým pracuji pouze výjimečně a mnoho konstrukcí
jazyka jsem si musel před zahájením tvorby programu připomenout. Samotnou kapitolu
lze pak vyčlenit pro síťovou komunikaci, jejíž implementace je v případě jazyka C
o poznání složitější záležitostí (ve srovnání s Javou).

\subsection{Síťová komunikace}
Již při analýze jsem zjistil, že obsluhu požadavků klientů bude možno v rámci
jazyka C řešit dvěma způsoby. Jedno z možných řešení spočívalo ve využití
příkazu fork() (vytvoření nového vlákna), jež by byl volán při každém připojení nového
klienta. Tento způsob se mi příliš nezamlouval, jelikož při větším počtu připojených klientů
by velice pravděpodobně vedl k velkému vytížení serveru (hodně vláken
běžících současně) a mohl by vést k nepřehlednosti během ladění programu.

Rozhodl jsem se tedy využít příkazu select(), jež nevyžaduje
vytváření dalšího vlákna při připojení nového klienta a umožňuje
obsluhu více klientů v rámci jednoho vlákna. Více informací viz kapitola
zabývající se implementací.

\chapter{Popis implementace (programátorská dokumentace)}
Jelikož během práce byl vytvářen herní server i klient (každý v jiném programovacím jazyce),
rozhodl jsem se popis implementace rozdělit do tří částí. V rámci první
části je popsán protokol, jež jsem pro hru vytvořil a pomocí kterého
si klient vyměňuje zprávy s herním serverem. Druhá sekce textu obsahuje popis implementace serveru - zde
je například popsán způsob ochrany serveru proti narušitelům atd. Další část textu pojednává o
klientské části aplikace.

\section{Komunikační protokol}
Při návrhu komunikačního protokolu jsem snažil snoubit bezpečnost s jednoduchostí
- validní zprávy mají jasně definovanou délku, ve většině případů i obsah.
Každá validní zpráva v rámci programu musí začínat znakem "\#" a končit
znakem "?".

\subsection{Rozdělení hry na fáze}
Hra je z pohledu serveru dělena do čtyř fází. Server si uchovává pro každého připojeného hráče strukturu, v rámci
níž jsou uvedeny informace o fázi hry, ve které se hráč nachází. Ve struktuře 
je uvedena i IP adresa hráče, která slouží jako jednoznačný identifikační
údaj.

\subsection{Ochrana vůči narušitelům}
Rozklad hry na jednotlivé fáze je také základní stavební kámen obranného mechanismu mého serveru.
Každá fáze hry má totiž jasně definované zprávy, které jsou v dané fázi validní.
Tuto skutečnost je možno rovněž chápat jako bezpečnostní opatření, jelikož pro každou fázi hry jsou jasně definované
validní zprávy, které mohou od klienta přijít. Pokud přijde od libovolného klienta 
zpráva, jež neodpovídá požadavkům (délka či obsah), pak dojde k odpojení daného klienta.

\subsection{Zprávy předávané v 1. fázi hry}
V první fázi hry se uživatel nachází, pokud klient
úspěšně navázal spojení se serverem. V této fázi
je za validní považován pouze jeden typ zpráv.

\subsubsection{Zpráva s délkou 17B - přezdívka}
Pokud klient v první fázi pošle zprávu o délce 17B,
je její obsah požadován za jméno hráče (15 znaků). 

Požadavky: jméno hráče musí být uvedeno bez diakritiky a nesmí obsahovat
znak procenta (procenta doplňuje klientská aplikace do délky 15 znaků).

Formát zprávy: "\#xxxxxxxxxxxxxxx?", kde x = písmeno obsažené v přezdívce uživatele

Odpověď serveru:
\begin{itemize}
\item SERVER\_WRITE\_RECEIVED\_OK ("\#99?") = server jméno přijal, úspěch => 2. fáze hry
\item SERVER\_WRITE\_RECEIVED\_FAIL ("\#98?") = zpráva od klienta je nevalidní (chybí počáteční nebo ukončovací znak), chyba
\end{itemize}

\subsection{Zprávy předávané ve 2. fázi hry}
V druhé fázi hry se uživatel nachází, pokud klient úspěšné prošel
první fází (tzn. připojil se k serveru, úspěšně
 navázal spojení se serverem a odeslal přezdívku). V druhé fázi
si uživatel typicky vybírá herní místnost, k níž se připojí.
Do této fáze se řadí i vytvoření nové herní místnosti.
V dané fázi jsou za validní považovány pouze tři typy zpráv.

\subsubsection{Zpráva s délkou 13B končící "0" - připojení k existující hře}
Pokud klient pošle zprávu o délce 13B, je považována
za validní a předpokládá se, že obsahuje název místnosti, ke které
se chce klient připojit (10 znaků) a za ní "0". Poslední
znak serveru říká, že hra by již měla být vytvořena a nemá se pokoušet
o její založení.

Požadavky: název herní místnosti musí být bez diakritiky a nesmí obsahovat
znak procenta.

Formát zprávy: "\#xxxxxxxxxx0?", kde x = písmeno obsažené v názvu místnosti

Odpověď serveru:
\begin{itemize}
\item SERVER\_WRITE\_RECEIVED\_OK ("\#99?") = hra existuje, uživatel připojen => 3. fáze hry
\item SERVER\_WRITE\_RECEIVED\_FAIL ("\#98?") = zpráva od klienta je nevalidní, chyba
\item SERVER\_STAGE2\_WRITE\_ROOM\_DOES\_NOT\_EXIST ("\#96?") = zvolená herní místnost neexistuje
\item SERVER\_STAGE2\_WRITE\_ROOM\_RUNNING ("\#95?") = ve zvolené místnosti již běží hra, jejíž součástí není daný uživatel
\end{itemize}

\subsubsection{Zpráva s délkou 13B končící "1" - vytvoření nové hry}
Pokud klient pošle zprávu o délce 13B, je považována
za validní a pokud končí "1", předpokládá se, že obsahem zprávy je název
místnosti, jež chce uživatel založit (10 znaků) a za ní "1". Poslední
znak značí, že chce uživatel vytvořit novou místnost (ne se připojit k existující).

Požadavky: název herní místnosti musí být bez diakritiky a nesmí obsahovat
znak procenta.

Formát zprávy: "\#xxxxxxxxxx1?", kde x = písmeno obsažené v názvu místnosti

Odpověď serveru:
\begin{itemize}
\item SERVER\_WRITE\_RECEIVED\_OK ("\#99?") = hra úspěšně založena => 3. fáze hry
\item SERVER\_WRITE\_RECEIVED\_FAIL ("\#98?") = zpráva od klienta je nevalidní, chyba
\item SERVER\_STAGE2\_WRITE\_ROOM\_EXISTS ("\#97?") = herní místnost s daným názvem již existuje, chyba
\end{itemize}

\subsubsection{Zpráva s délkou 3B ("\#1?") - získání seznamu herních místností}
Pokud klient pošle kód SERVER\_STAGE2\_GET\_GAME\_ROOM\_LIST ("\#1?"),
pak je to chápáno jako požadavek o odeslání seznamu existujících herních místností.

Formát zprávy: "\#1?"

Odpověď serveru:
\begin{itemize}
\item zpráva o délce 52B obsahující seznam herních místností (5 místností maximum * 10B pro název místnosti)
\end{itemize}

\subsection{Zprávy předávané ve 3. fázi hry}
Ve třetí fázi hry se uživatel nachází, pokud klient úspěšně prošel
předcházejícími fázemi (tzn. je připojen, odeslal své jméno a nachází se v některé z herních místností).
V této fázi uživatel čeká v herní místnosti na připojení dalšího hráče - má možnost hru odstartovat,
či se vrátit do herního lobby. Následuje seznam očekávaných zpráv a možných reakcí na ně.

\subsubsection{Zpráva s délkou 4B ("\#5?") - odstartování hry}
Pokud klient pošle kód SERVER\_STAGE3\_START\_GAME ("\#5?") je tato zpráva
požadována za validní požadavek o start hry uvnitř herní místnosti, v níž se daný hráč nachází.

Formát zprávy: "\#5?"

Odpověď serveru:
\begin{itemize}
\item SERVER\_WRITE\_RECEIVED\_OK ("\#99?") = požadavek na start hry přijat => 4. fáze hry
\item SERVER\_WRITE\_RECEIVED\_FAIL ("\#98?") = zpráva od klienta je nevalidní, chyba
\item SERVER\_STAGE3\_WRITE\_GAME\_CANNOT\_START ("\#11?") = v místnosti je pouze jeden hráč, chyba
\end{itemize}


\subsubsection{Zpráva s délkou 4B ("\#6?") - návrat do lobby}
Pokud klient pošle kód SERVER\_STAGE3\_BACK\_LOBBY ("\#6?") je tato zpráva
požadována za validní požadavek o návrat do lobby. Po zpracování
takové zprávy je klient přesunut zpět do fáze 2.

Formát zprávy: "\#6?"

Odpověď serveru:
\begin{itemize}
\item SERVER\_WRITE\_RECEIVED\_OK ("\#99?") = požadavek na přesun do lobby přijat
\end{itemize}

\subsubsection{Zpráva s délkou 4B ("\#7?") - zjištění stavu hry}
Pokud klient pošle kód SERVER\_STAGE3\_GAME\_STATE ("\#7?"),
dojde k vyhodnocení stavu hry uvnitř herní místnost (hra běží / neběží).
Tento kód je klienty odesílán v periodickém čas. intervalu a klienti tak zjišťují,
zda již byla hra odstartována, či nikoliv.

Formát zprávy: "\#7?"

Odpověď serveru:
\begin{itemize}
\item SERVER\_WRITE\_RECEIVED\_OK ("\#99?") = server obdržel zprávu, hra neběží
\item SERVER\_STAGE3\_WRITE\_GAME\_STARTED ("\#10?") = server obdržel zprávu, hra byla spuštěna => 3. fáze hry
\end{itemize}

\subsection{Zprávy předávané ve 4. fázi hry}
Ve čtvrté fázi hry se uživatel nachází, pokud klient prošel všemi předcházejícími fázemi (tzn. připojen,
odeslal přezdívku, vybral místnost a někdo odstartoval hru). Takový hráč se tedy typicky nachází uvnitř některé z herních místností
a jsou od něj očekávány následující zprávy.

\subsubsection{Zpráva s délkou 4B ("\#50?") - získání jmen hráčů}
Na požadavek SERVER\_STAGE4\_GET\_PLAYER\_NAMES ("\#50?") server
zareaguje odesláním přezdívek hráčů, jež se nacházejí v herní místnosti společně
s hráčem, který tuto zprávu odeslal.

Formát zprávy: "\#50?"

Odpověď serveru:
\begin{itemize}
\item zpráva o délce 47B obsahující jména hráčů, jež se nacházejí v dané herní místnosti (3 hráči maximum * 15B pro jednu přezdívku)
\end{itemize}

\subsubsection{Zpráva s délkou 4B ("\#51?") - získání přidělených karet}
Při příchodu kódu SERVER\_STAGE4\_GET\_CARDS ("\#51?") server
odešle seznam karet, které má uživatel přidělen. Pokud zpráva přijde od hráče,
který se do místnosti nově připojil, předchází odeslání seznamu karet ještě přidělení náhodných karet uživateli.

Seznam obsahuje pouze "0" (pokud karta, jež odpovídá danému indexu nebyla přidělena) a "1" (pokud byla přidělena). Indexy pole s kartami jsou shodné pro obě strany - server i klient. Indexy pole karet - viz další kapitola.

Formát zprávy: "\#51?"

Odpověď serveru:
\begin{itemize}
\item zpráva o délce 34B reprezentující karty přidělené uživateli
\end{itemize}

\subsubsection{Zpráva s délkou 4B ("\#01?" - "\#44?") - použití přidělené karty}
Kódy v rozmezí "\#01?" - "\#44?" jsou chápány jako identifikační kódy karet a příchod takového
kódu je ze strany serveru chápán jako pokus o použití karty ve hře.

Kódy "\#01?" - "\#32?" jsou přiřazeny běžným kartám (32 karet v balíčku), zbývající kódy
jsou pak v rámci aplikace vyhrazeny pro další potřebné informace o kartě, kterou chce
uživatel umístit. Kódy "\#33?" - "\#44?" jsou tedy ze strany klienta odeslány, pokud
chce uživatel umístit kartu, jejíž efekt nemusí být ze vzhledu karty patrný => jedná se o "měniče".
Uživatel může chtít umístit například červeného měniče a přitom změnit barvu
na zelenou - v takovém připadě by byl např. odeslán kód "\#37?". Kompletní
seznam kódu karet je dostupný níže.

Nejprve je zkontrolována platnost daného úkonu (zejména zjištění, zda je možno
danou kartu umístit na předchozí). Následně se zkontroluje, zda je odesílatel na tahu.
Závěrem proběhne kontrola, zda uživatel danou kartu opravdu vlastnil (tzn. byla mu serverem přidělena a nejedná se o podvodníka).
Od výsledku těchto kontrol se odvíjí odpověď serveru.

Formát zprávy: "\#xx?", kde xx = číslo příslušné karty


Kódové označení karet:
\begin{itemize}
\item "(\#01?") = koule 7
\item "(\#02?") = koule 8
\item "(\#03?") = koule 9
\item "(\#04?") = koule 10
\item "(\#05?") = koule spodek
\item "(\#06?") = koule menic
\item "(\#07?") = koule kral
\item "(\#08?") = koule eso
\item "(\#09?") = cervena 7
\item "(\#10?") = cervena 8
\item "(\#11?") = cervena 9
\item "(\#12?") = cervena 10
\item "(\#13?") = cervena spodek
\item "(\#14?") = cervena menic
\item "(\#15?") = cervena kral
\item "(\#16?") = cervena eso
\item "(\#17?") = zelena 7
\item "(\#18?") = zelena 8
\item "(\#19?") = zelena 9
\item "(\#20?") = zelena 10
\item "(\#21?") = zelena spodek
\item "(\#22?") = zelena menic
\item "(\#23?") = zelena kral
\item "(\#24?") = zelena eso
\item "(\#25?") = zaludy 7
\item "(\#26?") = zaludy 8
\item "(\#27?") = zaludy 9
\item "(\#28?") = zaludy 10
\item "(\#29?") = zaludy spodek
\item "(\#30?") = zaludy menic
\item "(\#31?") = zaludy kral
\item "(\#32?") = zaludy eso
\item "(\#33?") = koule měnič => červená
\item "(\#34?") = koule měnič => zelená
\item "(\#35?") = koule měnič => žaludy
\item "(\#36?") = červená měnič => koule
\item "(\#37?") = červená měnič => zelená
\item "(\#38?") = červená měnič => žaludy
\item "(\#39?") = zelená měnič => koule
\item "(\#40?") = zelená měnič => červená
\item "(\#41?") = zelená měnič => žaludy
\item "(\#42?") = žaludy měnič => koule
\item "(\#43?") = žaludy měnič => červená
\item "(\#44?") = žaludy měnič => zelená
\end{itemize}

Odpověď serveru:
\begin{itemize}
\item SERVER\_WRITE\_RECEIVED\_OK ("\#99?") = karta úspěšně umístěna
\item SERVER\_WRITE\_RECEIVED\_FAIL ("\#98?") = zpráva od klienta je nevalidní, chyba
\item SERVER\_STAGE4\_WRITE\_CANNOT\_PLACE ("\#69?") = daná karta nemůže být umístěna na předchozí, chyba
\item SERVER\_STAGE4\_WRITE\_NOT\_TURN\_PLACE ("\#68?") = odesílatel není na tahu, chyba
\item SERVER\_STAGE4\_MUST\_TAKE\_1\_CARDS ("\#67?") = uživatel si musí vzít 1 kartu (umístěna karta 7), chyba
\item SERVER\_STAGE4\_MUST\_TAKE\_2\_CARDS ("\#66?") = uživatel si musí vzít 2 karty (umístěna karta 7), chyba
\item SERVER\_STAGE4\_MUST\_TAKE\_3\_CARDS ("\#65?") = uživatel si musí vzít 3 karty (umístěna karta 7), chyba
\item SERVER\_STAGE4\_MUST\_TAKE\_4\_CARDS ("\#64?") = uživatel si musí vzít 4 karty (umístěna karta 7), chyba
\item SERVER\_STAGE4\_MUST\_TAKE\_5\_CARDS ("\#63?") = uživatel si musí vzít 5 karet (umístěna karta 7), chyba
\item SERVER\_STAGE4\_MUST\_TAKE\_6\_CARDS ("\#62?") = uživatel si musí vzít 6 karet (umístěna karta 7), chyba
\item SERVER\_STAGE4\_MUST\_TAKE\_7\_CARDS ("\#61?") = uživatel si musí vzít 7 karet (umístěna karta 7), chyba
\item odpojení klienta od serveru = pokud použita jakákoliv karta, jež uživatel nemá přidělenou -> pokus o hack serveru
\end{itemize}

\subsubsection{Zpráva s délkou 4B ("\#45?") - vynechání kola / žádost o kartu}
Pokud přijde kód SERVER\_STAGE4\_TAKE\_STAND ("\#45?") server jej
vyhodnotí jako:
\begin{itemize}
\item a) žádost o vynechání kola - pokud je aktuální kartou eso, uživatel "stojí"
\item b) žádost o novou kartu - uživateli přidělena další karta z balíčku dostupných karet
\end{itemize}

Formát zprávy: "\#45?"

Odpověď serveru:
\begin{itemize}
\item kód karty ("\#0?" - "\#31?") = karta přidělena, pořadí přidělené karty v seznamu kódů karet, úspěch
\item SERVER\_WRITE\_RECEIVED\_FAIL ("\#98?") = zpráva od klienta je nevalidní, chyba
\item SERVER\_STAGE4\_WRITE\_NOT\_TURN\_TAKE ("\#79?") = karta nebyla přidělena, uživatel není na tahu, chyba
\item SERVER\_STAGE4\_WRITE\_NO\_FREE\_CARDS ("\#78?") = karta nebyla přidělena, v balíčku nejsou žádné volné karty, chyba
\end{itemize}

\subsubsection{Zpráva s délkou 3B ("\#0?") - žádost o stav hry}
Na kód SERVER\_PING\_CHAR ("\#0?") server reaguje odesláním zprávy velikosti 6B ve formátu XYYZ,
kde:

\begin{itemize}
\item X = pořadí hráče, jež je aktuálně na tahu (v rámci herní místnosti)
\item YY = kódové označení aktuální karty ("\#01?" - "\#44?") NEBO "\#45?" - pokud uživatel "stojí" / bere si další kartu
\item Z = pokud umístěná karta "platí" na dalšího uživatele (např. eso) - "\#1?" pokud aktivní, jinak "\#0?"
\end{itemize}

Zasílání tohoto kódu je klientem prováděno periodicky, aby byl vždy obeznámen s aktuálním stavem hry.

Formát zprávy: "\#0?" 

Odpověď serveru:
\begin{itemize}
\item zpráva velikosti 6B, specifikace viz výše 
\end{itemize}

\subsubsection{Zpráva s délkou 4B ("\#52?") - žádost o návrat do lobby}
Pokud klient pošle kód SERVER\_STAGE4\_BACK\_LOBBY ("\#52?") je tato zpráva
požadována za validní požadavek o návrat do lobby. Po zpracování
takové zprávy je klient přesunut zpět do fáze 2.

Formát zprávy: "\#52?"

Odpověď serveru:
\begin{itemize}
\item SERVER\_WRITE\_RECEIVED\_OK ("\#99?") = požadavek na přesun do lobby přijat => 2. fáze hry
\item SERVER\_WRITE\_RECEIVED\_FAIL ("\#98?") = zpráva od klienta je nevalidní, chyba
\end{itemize}

\section{Server}
\subsection{Technologie využité během vývoje}
Serverová část aplikace byla vytvořena v nízkoúrovňovém jazyce C.
Vývoj probíhal zejména ve vývojovém prostředí CLion, jež bylo spuštěno v rámci Linuxového operačního
systému Ubuntu.

\subsection{Požadavky pro spuštění}
Vzhledem k volbě jazyka je pro překlad programu nutno mít
na cílovém zařízení, na němž má být server spuštěn, nainstalovano GCC či jiný překladač
schopný přeložit zdrojové soubory jazyka C. Požadavky na výkon procesoru
a velikost paměti RAM jsou v případě mojí aplikace zanedbatelné, 
s jejím spuštěním by neměl mít problém žádný moderní PC. Je však
nutno poznamenat, že požadavky na výkon rostou s počtem
připojených klientů a pokud bychom chtěli server reálně nasadit, musíme počítat
s obsluhou tisíců klientů a výkonný hardware by byl v takovém případě
nevyhnutelný. Testování serveru probíhalo zejména na následující konfiguraci: 
12GB RAM, i5-5200U (mobilní CPU), Ubuntu. Server byl při testování na daném stroji stabilní
a nevykazoval známky zahlcení ze strany klientů (rychlé reakce na zprávy klientů atp.).

\subsection{Reakce na zprávy od klienta}
Informace o aktuální fázi je pro server důležitá,
jelikož od aktuální fáze hráče se odvíjí reakce serveru
na zprávu daného uživatele. Pokud je daná zpráva validní, 
server patřičně zareaguje (dle protokolu - viz výše).
V případě nevalidní zprávy je reakce predikovatelná, klient je zkrátka odpojen.

\subsection{Popis struktur použitých v rámci aplikace}
Zdrojový kód serverové části aplikace disponuje dostatečným množstvím komentářů a proto nepovažuji za
vhodné popisovat jednotlivé struktury v rámci této dokumentace.


\section{Klient}
\subsection{Technologie využité během vývoje}
Klient byl (dle požadavků) vytvořen ve vysokoúrovňovém jazyce Java.
Vývoj probíhal zejména ve vývojovém prostředí IntelliJ IDEA, jež bylo spuštěno v rámci Linuxového operačního
systému Ubuntu.

\subsection{Požadavky pro spuštění}
Na cílovém zařízení je potřeba mít nainstalovanou Javu minimálně ve verzi 11. Je možno použít verzi od firmy Oracle, či volně dostupnou OpenJDK. Systémové
požadavky klienta jsou minimální a v dnešní době s během klienta nebude mít problém prakticky žádné zařízení. Nejslabší PC, na kterém
byl klient testován disponovalo následující konfigurací: Intel Atom z3735f, 1GB RAM. Jelikož jsem během provozování klienta
na zmíněné konfiguraci nepozorval žádné problémy, troufám si tvrdit, že je poměrně dobře optimalizovaný.

Rychlost odezvy klienta je dána zejména tím, že téměr veškeré výpočty, jež program koná (náhodný výběr karet, atp.), jsou vykonávány na straně serveru.

\subsection{Zasílání zpráv serveru}
Použití mnou vytvořeného klienta hráči zaručí, že na server budou odeslány pouze validní zprávy a při hraní by tak neměl narazit na žádný problém.
Zasílány jsou zprávy z protokolu (viz výše). I přes tuto skutečnost jsem se serverem zkoušel komunikovat prostřednictvím terminálu a při jeho použití
jsem nenarazil na žádný problém.

\subsection{Popis struktur použitých v rámci aplikace}
Klientská část aplikace, podobně jako server, je opatřena mnoha komentáři (javadoc) a zde tedy nebudu strukturu aplikace rozepisovat.

\chapter{Uživatelská dokumentace}
Níže je popsán postup, který je nutno provést pro přeložení a následné spuštění serverové i klientské části aplikace (z pohledu uživatele).

\section{Pravidla hry}
Vzhledem k faktu, že jsem vytvořil hru prší s klasickými pravidly, níže jmenuji pouze základní. Kompletní pravidla lze v případě potřeby nalézt online, např. \href{https://www.tipmag.cz/navod-na-hru-prsi}{zde}.

Pravidla:
\begin{itemize}
\item na začátku hry každý hráč dostane čtyři karty
\item každá karta má určitou hodnotu a barvu, na kartu lze položit pouze kartu se stejnou hodnotou / barvou
\item pokud hráč nemá kartu, jež by byl schopen vyložit a nemůže "stát" (viz dále), pak si musí vzít další kartu
\item hráč použije kartu 7 = další hráč musí dát 7 NEBO bere dvě karty
\item hráč použije eso = další hráč musí dát eso NEBO "stojí" (nebere si kartu, hraje další v pořadí)
\item hráč použije měniče = další hráč musí dát měniče NEBO kartu s barvou, na níž měnič změnil
\item hra končí, pokud jeden z hráčů již nemá žádné karty - takový hráč je vítěz
\end{itemize}

\section{Herní klient}
K překladu a spuštění je potřeba mít nainstalovanou Javu ve verzi 11. Překlad lze provést příkazem "make", jež bude proveden v rámci adresáře s klientskou částí aplikace.
Následné spuštění pak lze provést pomocí příkazu "make run".

\section{Server}
K překladu ja potřeba mít nainstalovaný překladač jazyka C. Překlad lze provést příkazem "make", jež bude proveden v rámci adresáře se serverovou částí aplikace.
Následné spuštění pak lze provést pomocí příkazu "./gameserver xxxxx", kde x = číslo portu, na kterém server poběží.

\chapter{Závěr}
\section{Funkčnost programu a jeho přínos}
Funkčnost programu jsem ověřil a pracuje dle očekávání. Jednalo se o první semestrální práci, v rámci níž jsem řešil síťovou komunikaci. Díky tomuto faktu
došlo k rozšíření mých znalostí v rámci jazyka Java i C.

\section{Zhodnocení zadání}
Zadání úlohy se mi zpočátku zdálo velice triviální, ale nakonec se ukázalo, že naprogramování této úlohy vyžaduje nastudování několika algoritmů, z nichž některé jsou,
minimálně pro začátečníka v jazyce C, poměrně náročné na implementaci.
Po prostudování algoritmů (spojených zejména se síť. komunikací) a vyřešení několika drobných problémů se mi však úlohu podařilo vyřešit.

\end{document}